\section*{Chapter 5 - Syntax and
Semantics}
\subsection*{5.1 Introduction}
Further thoughts about semantics are explored in chapter 5.14 and maybe the notation of structures here will be clearer once one has worked through the next sub-chapters. The point of semantics is to talk about {\it truth} of formulas, whereas syntax only concerns the construction of formulas.\footnote{A pedagogical explanation can be found in  \url{https://leanprover.github.io/logic_and_proof/semantics_of_first_order_logic.html}} 
To clarify the notation of a structure, a structure for our exploration in first order logic contains three parts:
\begin{itemize}
    \item Our domain which is the set of atomics
    \item The signature which is all functions and relations
    \item Lastly the interpretations, in this case it is a satisfaction relation on the previous, which is our truth assignment 
\end{itemize} 
Our first order structures will be working with an interpretation called the satisfaction relation, explained as "A formula may be satisfied in a structure M relative to a variable assignment s, written as M,s $\models$ A." Where formula means a term and a predicate symbol. If the satisfaction is independent of the variable assignment, then we say that the formula is satisified in the structure. Such a formula independent of variable assignment is called a sentence.

Our satisfaction relation is therefore our interpretation of what is true or false, what is satisfied. When the notion of validity is introduced we see that some sentences are true in every model we build.